\section{LOTHAIRE, ALGEBRAIC, ROZDZIAŁ 1}

Strona 12-14: definicja automatu
Definicja: Zbiór $X \subset \A^*$ jest rozpoznawalny, jeśli istnieje skończony automat $A$ taki, że istnieje relacja \ldots

Strona 23-25: funkcje tworzące

\section{LOTHAIRE, ALGEBRAIC, ROZDZIAŁ 2}

\subsection{Słowa Sturmowskie}

Definicja: słowo $w$ jest Sturmowskie jeśli dla każdego $n \ge 0$ zachodzi $|F_n(w)| = n + 1$.

Problem: słowo $w$ jest Sturmowskie wtedy i tylko wtedy, dla każdego $n$ istnieje dokładnie jedno podsłowo $u$ z $|u| = n$ takie, że $u0$ i $u1$ są podsłowami $w$.

Problem: sufiks słowa Sturmowskiego też jest Sturmowski.

Problem [2.1.1, 2.1.4, s. 47-48]: słowa Fibonacciego są Sturmowskie

Definicja: wysokość słowa $h(w) = |\{i: w[i] = 1\}|$

Definicja: równowaga słowa $\delta(v, w) = |h(v) - h(w)|$

Definicja: zbiór $X$ jest zrównoważony jeśli dla każdych $v, w \in X$ z $|v| = |w|$ zachodzi $\delta(v, w) \le 1$.

Problem [2.1.3, s. 48-49]: zbiór $X$ zamknięty ze względu na podsłowa nie jest zrównoważony wtedy i tylko wtedy, gdy istnieje palindrom $w$ taki, że $0w0$ i $1w1$ \ldots

Problem [2.1.19, s. 61]: zbiór słów Sturmowskich jest domknięty ze względu na odwracanie.

Problem: dla każdego słowa Sturmowskiego s albo 0s albo 1s jest Sturmowskie.

\subsection{Słowa centralne i standardowe}

Całe omówienie: [s. 64-81]

Definicja: Słowo $w$ jest centralne, jeśli $w \in 0^* \cup 1^* \cup (P \cap P10P)$ dla zbioru wszystkich palindromów $P$.

Faktoryzacja $w = p10p$ jest unikalna.

$P \cap P10P = P \cap P01P$

Definicja: Słowo $w$ jest centralne jeśli $w01$ jest standardowe.
Równoważnie: $w$ jest centralne jeśli $w10$ jest standardowe.
Wniosek: jeśli prefiks lub sufiks słowa centralnego jest palindromem, to jest centralny [prefiks i sufiks?]

$(0, 1)$ to para standardowa. Jeśli $x(x, y)$ to para standardowa, to $L(x, y) = (x, xy)$ i $R(x, y) = (yx, y)$ też.
Słowo standardowe to słowo, które należy do jakiejś pary standardowej.

Problem: Pokazać, że słowa Fibonacciego są standardowe.
Dowód: $(f_{2n + 2}, f_{2n + 1}) = RL(f_{2n}, f_{2n - 1})$ [+/- przesunięcie o 1]

\section{LOTHAIRE, ALGEBRAIC, ROZDZIAŁ 8}

\subsection{Złoty podział}

[8.1.12, s. 279]
Jeśli $v$, $w \in A^+$ i $\rho, \sigma \in \Q_+$, $\phi < \rho < \sigma$, takie, że $v^\rho = w^\sigma$, to istnieje $u \in A^+$ i $\tau \in \Q_+$, $\tau \ge \rho + 1$ takie, że $u^\tau$ jest prefikso-sufiksem $v^\rho$.

[8.1.13, s. 280]
Problem: Pokazać, że liczby $\phi$ nie da się poprawić.

%%%%%%%%%%%%%%%%%%%%%%%%%%%%%%%%%%

\section{Najmniejsze słowo pokrywające}

\begin{algorithm}[H]
    \caption{Najmniejsze słowo pokrywające}
    \Input{Słowo $t \in \A^+$ ($|t| = n$)}
    \Output{Najkrótsze możliwe słowo $w \in \A^+$ takie, że dla każdego $1 \le i \le n$ istnieje $1 \le j \le m$ spełniające $t[i] = w[j]$.}
\end{algorithm}

% Najdłuższe wspólne podsłowo

Dla ustalonego $k$ istnieje prosty algorytm o złożoności $O(n)$: wystarczy wypróbować wszystkie możliwe permutacje kolejności wystąpień słów w szukanym nadsłowie i wykonać łączenie od lewej do prawej zachłannie, uzgadniając jak największy sufiks poprzedniego słowa z jak największym prefiksem następnego.

\subsection{Algorytm na bazie pokrycia zbioru}

\begin{algorithm}
\caption{Algorytm 2$H_n$-przybliżony}
\label{2Hn-APX_shortest_superstring}
\begin{algorithmic}
\State Dla każdej pary słów $s_i, s_j \in S$ i $k > 0$ stwórz wszystkie możliwe słowa $\sigma_{ijk}$ powstałe przez nałożenie wspólnego $k$-literowego fragmentu $s_i$ i $s_j$. Niech $M$ będzie zbiorem wszystkich takich słów.
\State Wykonaj algorytm \ref{Hn-APX_set_cover} dla uniwersum $S$ i jego podzbiorów $set(\pi) = \{p \in S: p \sqsubseteq \pi\}$ dla $\pi \in S \cup M$ o funkcji kosztu $c(set(\pi)) = |\pi|$.
\State $\Pi = \{set(\pi_1), set(\pi_2), ... set(\pi_k)\}$ jest zbiorem zwróconym przez algorytm \ref{Hn-APX_set_cover}.
\Return $\pi_1 \circ \pi_2 \circ ... \circ \pi_k$
\end{algorithmic}
\end{algorithm}

\begin{theorem}
Algorytm \ref{2Hn-APX_shortest_superstring} jest $2H_n$-przybliżony dla problemu najkrótszego nadsłowa.
\end{theorem}
\begin{proof}
Dowód sprowadza się do przedstawienia szeregu faktów:
\begin{itemize}
\item Skoro $\Pi$ jest jest pokryciem zbiorami dla uniwersum $S$, to znaczy, że dla każdego $p \in S$ istnieje pewien $set(\pi_i) \in \Pi$ taki, że $p \in set(\pi_i)$, czyli $p \sqsubseteq \pi_i$. To oznacza, ze zwrócone przez algorytm rozwiązanie jest dopuszczalne: $APX \geq OPT$.
\item Zgodnie z lematem \ref{superstring_cover} optymalne rozwiązanie tak sformułowanego problemu pokrycia zbiorami jest równoważne otrzymaniu wyniku 2-przybliżony dla problemu najkrótszego nadsłowa.
\item A ponieważ rozwiązanie dla problemu pokrycia zbiorami zwrócone przez algorytm \ref{Hn-APX_set_cover} jest co najwyżej $H_n$ razy większe od optymalnego, to otrzymane rozwiązanie będzie spełniało zależność $APX \leq 2H_n \cdot OPT$.
\end{itemize}
\end{proof}

\begin{lemma}
\label{superstring_cover}
Optymalne rozwiązanie problemu pokrycia zbioru dla danej instancji: uniwersum $S$, jego podzbiorów $set(\pi)$ i funkcji kosztu $c(set(\pi))$ jest rozwiązaniem 2-przybliżonym dla problemu najkrótszego nadsłowa słów z $S$.
\end{lemma}
\begin{proof}
Uszeregujmy słowa $\{s_1, s_2, ..., s_n\}$ zgodnie z ich kolejnością występowania w optymalnym rozwiązaniu $s$. Wówczas istnieje pokrycie zbiorami w ten sposób, że $\pi_1$ jest nadsłowem słów od $s_1$ do możliwie największego $s_i$ takiego, że $s_1$ i $s_i$ nakładają się w $s$. Analogicznie $\pi_2$ jest nadsłowem słów od $s_{i+1}$ do możliwie największego $s_j$ takiego, że $s_{i+1}$ i $s_j$ nakładają się w $s$ itd. \\
W tak zbudowanym pokryciu zachodzi $\pi_i \in S \cup M$, czyli pokrycie jest dopuszczalne. \\
Oznaczmy, że $\pi_i$ kończy się słowem $s_j$, $\pi_{i+1}$ zaczyna się od $s_{j+1}$ i kończy na $s_k$, a $\pi_{i+2}$ zaczyna się od $s_{k+1}$. Ponieważ $s_{k+1}$ nie nakłada się z $s_{j+1}$ w $s$ (należałoby wtedy do $\pi_{i+1}$, a nie $\pi_{i+2}$). Tym bardziej nie może nakładać się z $s_j$, a co za tym idzie żadne dwa słowa $\pi_i$ i $\pi_{i+2}$ nie nakładają się w $s$. \\
Każda litera $s$ pokryta jest za pomocą co najwyżej dwóch słów z $\Pi$, czyli pokazane rozwiązanie $\pi_1 \circ \pi_2 \circ ... \circ \pi_k$ jest co najwyżej 2 razy większe niż optymalne $s$.
\end{proof}

\begin{algorithm}
\caption{}
\label{4-APX_shortest_superstring}
\begin{algorithmic}
\State Skonstruuj graf prefiksowy $G$ odpowiadający słowom z $S$.
\State Znajdź w $G$ pokrycie cyklowe $C = \{c_1, ..., c_k\}$ o najmniejszej wadze.
\Return $\sigma(c_1) \circ ... \circ \sigma(c_k)$
\end{algorithmic}
\end{algorithm}

\begin{remark}
Oznaczenia dla cyklu $c_i$:
\begin{itemize}
\setlength{\itemsep}{1pt}
\setlength{\parskip}{0pt}
\setlength{\parsep}{0pt}
\item $\alpha(c_i) = prefix(s_{i_1}, s_{i_2}) \circ ... \circ prefix(s_{i_l}, s_{i_1})$
\item $\sigma(c_i) = \alpha(c_i) \circ s_{i_1}$
\item $r_i = s_{i_1}$
\item $w(c_i) = |\alpha(c_i)|$
\item $w(C) = \sum\limits_{i = 1}^k w(c_i)$
\end{itemize}
\end{remark}

\begin{theorem}
Algorytm \ref{4-APX_shortest_superstring} jest 4-przybliżony dla problemu najkrótszego nadsłowa.
\end{theorem}
\begin{proof}
Dowód sprowadza się do przedstawienia szeregu faktów:
\begin{itemize}
\item Ponieważ rozwiązanie optymalne jest ograniczone z dołu przez długość najkrótszej trasy Hamiltona w $G$ to tym bardziej przez długość najkrótszego pokrycia cyklowego: $w(C) \leq w(C_H) \leq OPT$
\item Najkrótsze nadsłowo słów $r_i$ (ponumerowanych zgodnie z kolejnością ich wystąpień w najkrótszym nadsłowie $s$) jest krótsze niż najkrótsze nadsłowo słów $s_i$: $OPT \geq \sum\limits_{i = 1}^k r_i - \sum\limits_{i = 1}^{k - 1} overlap(r_i, r_{i + 1})$. Korzystając z lematu \ref{overlap} mamy $\sum\limits_{i = 1}^k r_i \leq OPT + \sum\limits_{i = 1}^{k - 1} overlap(r_i, r_{i + 1}) \leq OPT + \sum\limits_{i = 1}^k w(c_i) \leq 3 \cdot OPT$.
\item Algorytm zwraca rozwiązanie $APX = \sum\limits_{i = 1}^k |\sigma(c_i)| = w(C) + \sum\limits_{i = 1}^k r_i \leq OPT + 3 \cdot OPT = 4 \cdot OPT$.
\end{itemize}
\end{proof}

\begin{lemma}
\label{cycle}
Jeśli dla pewnego słowa $s$ wszystkie słowa z $S$ są podsłowami $s^\infty$, to w grafie prefiksowym istnieje cykl o wadze co najwyżej $|s|$ przechodzący przez wszystkie wierzchołki odpowiadające słowom z $S$.
\end{lemma}
\begin{proof}
Wszystkie słowa z $S$ zaczynają się w pierwszej kopii słowa $s$ i to w różnych miejscach tego słowa $s$, bo dla żadnych $p, q \in S$ $p \sqsubseteq q$. Jeśli cykl będzie odwiedzał wierzchołki w kolejności wystąpień ich początków w $s$, to waga tego cyklu będzie nie większa niż $|s|$.
\end{proof}

\begin{lemma}
\label{overlap}
Dla dowolnej pary cykli $c_1, c_2 \in C$ zachodzi $overlap(r_1, r_2) > w(c_1) + w(c_2)$.
\end{lemma}
\begin{proof}
Po kolei można zaobserwować:
\begin{itemize}
\item $s_{i_1}, s_{i_2}, ..., s_{i_l} \sqsubseteq \alpha(c_i)^\infty$
\item Skoro $r_1 \sqsubseteq \alpha(c_1)^\infty$, to $overlap(r_1, r_2) \sqsubseteq \beta(c_1)^\infty$ dla $\beta(c_1) \sqsubseteq overlap(r_1, r_2)$ i $|\beta(c_1)| = w(c_1)$.
\item Analogicznie skoro $r_2 \sqsubseteq \alpha(c_2)^\infty$, to $overlap(r_1, r_2) \sqsubseteq \beta(c_2)^\infty$ dla $\beta(c_2) \sqsubseteq overlap(r_1, r_2)$ i $|\beta(c_2)| = w(c_2)$.
\item Jeśli $overlap(r_1, r_2) \geq w(c_1) + w(c_2)$, to zarówno $\beta(c_1) \circ \beta(c_2) \sqsubseteq overlap(r_1, r_2)$, jak i $\beta(c_2) \circ \beta(c_1) \sqsubseteq overlap(r_1, r_2)$.
\item To oznaczałoby, że $\beta(c_1)^\infty = \beta(c_2)^\infty$, zatem z lematu \ref{cycle} istniałby cykl o wadze $w(c_1)$ przechodzący wszystkie wierzchołki cykli $c_1$ i $c_2$, co przeczy minimalności pokrycia cyklowego $C$.
\end{itemize}
\end{proof}

\section{Ważone pokrycie wierzchołkowe jako ważone pokrycie zbiorami}

$\sigma_{ijk}$ -- słowo powstałe z nałożenia na siebie $k$ ostatnich liter słowa $s_i$ z pierwszymi $k$ literami słowa $s_j$
$M = \{\sigma_{ijk}\}$
$\set(\pi)$ -- zbiór wszystkich słów z $S$ takich, że $s$ jest podsłowem $\pi$

$U = S$, $\mathfrak{S} = \{\set(\pi)\colon \pi \in S \cup M\}$, $c(set(\pi)) = |\pi|$

Wykonaj algorytm (optymalny) dla problemu ważonego pokrycia zbiorami dla $(U, \mathfrak{S}, c)$
Rozwiązaniem jest $\set(\pi_1)$, \ldots
\return $s = \pi_1 | \pi_2 | \ldots$

\begin{theorem}
	Jeśli rozwiązanie problemu pokrycia zbioru jest optymalne, to algorytm jest $2$-przybliżony dla problemu najkrótszego wspólnego nadsłowa
\end{theorem}

\begin{proof}
	Każde słowo $s_i \in S$ jest pokryte przez pewne słowo $\pi_i$ z pokrycia zbiorami. Słowo $s_{APX}$ jest nadsłowem $\pi_1$, ldots, a więc odpowiada pokryciu zbiorami $\set(\pi_1)$, \ldots. 
	
	Istnieje pokrycie zbiorami o wadze $\le 2 OPT$.
	
	Jeśli uszeregujemy słowa $s_i$ wg ich występowania w $s_{OPT}$, to można wyznaczyć pary słów $(s_i, s_j)$ takie, że $s_i$ i $s_j$ nakładają się w $s$. Idąc od lewej do prawej: $(s_1, s_{i1 - 1})$, $(s_{i1}, s_{i2 - 1})$, $(s_{i2}, s_{i3 - 1})$, \ldots. Dla każdej pary istnieje odpowiednie słowo $\pi_i = \sigma_{1 {i1 - 1} k1}$, \ldots. Co więcej, słowo $\pi_i$ i $\pi_{i + 2}$ są rozłączne, bo jedno się kończy na $s_{i1 - 1}$, a drugie zaczyna na $s_{i2}$ -- a one są rozłączne, bo $s_{i1}$ i $s_{i2}$ są rozłączne.
	
	Wniosek: każda litera w $s_{OPT}$ jest pokryta przez co najwyżej dwa słowa $\pi_i$, a więc $APX = |s_{APX}| \le \sum |\pi_i| \le 2 |s_{OPT}| = 2 OPT$. 
\end{proof}

\begin{corollary}
	Algorytm jest $2 H_n$-przybliżony dla problemu najkrótszego wspólnego nadsłowa
\end{corollary}

\section{Algorytm $4$-przybliżony}

$overlap(s_i, s_j)$ -- najdłuższy sufiks $s_i$, będący prefiksem $s_j$
$prefix(s_i, s_j) = s_i - overlap(s_i, s_j)$

$OPT = \sum_{i = 1}^{m} prefix(s_i, s_{i + 1}) + overlap(s_n, s_1)$

Graf prefiksowy $K_n$, $w(v_iv_j) = |prefix(s_i, s_j)|$
	$OPT = |s_{OPT}| \ge \sum_{i = 1}^{m} |prefix(s_i, s_{i + 1})| = w(C_H) \ge w(C_P)$
Pokrycie cyklowe w G o minimalnej wadze = doskonałe skojarzenie w G' o minimalnej wadze
	wystarczy każdy wierzchołek v_i rozdzielić na dwóch niepołączonych ze sobą braci

Algorytm
	skonstruuj graf prefiksowy G
	znajdź najtańsze pokrycie cyklowe \mathbb{C} w G
	dla każdego cyklu $c_i \in \mathbb{C}$
		$\alpha(c_i) = prefix(s_i_1, s_i_2) | prefix(s_i_2, s_i_3) | \ldots prefix(s_i_n, s_i_1)$
		$\sigma(c_i) = \alpha(c_i) | s_i_1$ ($s_i_1$ jest reprezentantem cyklu $c_i$)
	return $\sigma(c_1) | \sigma(c_2) | \ldots$
	
\begin{lemma}
	Jeśli wszystkie słowa z $S' \subset S$ są podsłowami pewnego $t^\infty$, to w grafie prefiksowym $G$ istnieje cykl o wadze nie większej niż $|t|$ przechodzący przez wierzchołki odpowiadające wszystkim słowom z $S'$.
\end{lemma}

\begin{proof}
	Wystarczy wziąć słowa z $S'$ w kolejności wystąpień ich pierwszych liter w $t^\infty$ -- zaczynają się w różnych miejscach, ale wszystkie w pierwszej kopii słowa $t$.
\end{proof}

\begin{lemma}
	Jeśli $c$ i $c'$ są cyklami w $\mathbb{C}$, a $r$ i $r'$ ich reprezentantami, to $overlap(r, r') < |\alpha(c)| + |\alpha(c')|$
\end{lemma}

\begin{proof}
	Ponieważ $c_i$ jest cyklem pokrywającym $r$, to $r$ jest podsłowem $\alpha(c)^\infty$, a więc jest cykliczne co $|\alpha(c)|$. Podobnie, $r'$ -- co $|\alpha(c')|$.
	
	Zatem $overlap(r, r')$, jako podsłowo $r$ i $r'$, jest cykliczne jednocześnie co $|\alpha(c)|$ i co $|\alpha(c')|$. Niech $a$ i $a'$ oznaczają prefiksy $overlap(r, r')$ długości $|\alpha(c)|$ i $|\alpha(c')|$.
	
	Gdyby $overlap(r, r') \ge |\alpha(c)| + |\alpha(c')|$, to $a | a' = a' | a$, a więc $a | a' = a' | a$. Zatem dowolny prefiks $a^\infty$ jest zarazem prefiksem $(a')^\infty$, czyli $a^\infty = (a')^\infty$. Ale wówczas w $G$ istniałby cykl o wadze $|\alpha(c)|$, który pokrywałby wierzchołki z $c$ i $c'$ -- sprzeczność.
\end{proof}

\begin{theorem}
	Algorytm jest $4$-przybliżony dla problemu najkrótszego wspólnego nadsłowa
\end{theorem}

\begin{proof}
	Każde słowo $s_i$ jest podsłowem pewnego $\sigma(c_j)$, ponieważ $\mathbb{C}$ jest pokryciem cyklowym w $G$. Zatem rozwiązanie jest dopuszczalne.
	
	Niech $r_1$, $r_2, \ldots będzie kolejnością zgodnie z występowaniem reprezentantów w optymalnym nadsłowie. Z Lematu 1 wynika, że $\sum |\alpha(c)| \le OPT$. Z Lematu 2 wynika, że $\sum overlap(r_i, r_{i + 1}) \le 2 \sum |\alpha(c)| \le 2 OPT$. Co więcej, $OPT \ge \sum |r_i| - \sum overlap(r_i, r_{i + 1}) \ge \sum |r_i| - 2 OPT$, więc $\sum |r_i| \le 3 OPT$
	
	$APX = \sum |\sigma(c_i)| = \sum |\alpha(c_i)| + \sum |r_i| \le OPT + 3 OPT = 4 OPT$.
\end{proof}

\section{Algorytm $3$-przybliżony}

Algorytm
	skonstruuj graf prefiksowy G dla S
	znajdź najtańsze pokrycie cyklowe \mathbb{C} w G
	dla każdego cyklu $c_i \in \mathbb{C}$
		$\alpha(c_i) = prefix(s_i_1, s_i_2) | prefix(s_i_2, s_i_3) | \ldots prefix(s_i_n, s_i_1)$
		$\sigma(c_i) = \alpha(c_i) | s_i_1$ ($s_i_1$ jest reprezentantem cyklu $c_i$)
	skonstruuj graf zakładek G' dla \{\sigma(c_i)\}
	znajdź najcięższe pokrycie cyklowe $\mathbb{C}'$ w $G'$
	z każdego cyklu z $\mathbb{C}'$ usuń najtańszą krawędź
	sklej słowa $\{\sigma(c_i)\}$ według pokrycia ścieżkami z $\mathbb{C}'$ w słowo $\sigma$
	return $\sigma$

\begin{lemma}
	Algorytm z grafem zakładek gwarantuje kompresję nie mniejszą niż połowa optymalnej dla dowolnego zbioru słów $S = \{\sigma_1, \ldots, \sigma_n\}$
\end{lemma}

\begin{proof}
	Niech $\sigma_{OPT}$ będzie optymalnym nadsłowem $S$. Wówczas $OPT' = \sum |\sigma_i| - |\sigma_{OPT}| = \sum_{i = 1}^{n - 1} overlap(\sigma_i, \sigma_{i + 1}) = w(P_H)$ -- czyli koszt ścieżki komiwojażera w grafie zakładek $G'$ dla $S$.
	
	Z drugiej strony, każdy cykl ma co najmniej 2 krawędzie, zatem $APX' = w(P_P) \ge 0.5 w(C_P) \ge 0.5 w(C_H) \ge 0.5 w(P_H)$. Zatem $APX \ge 0.5 OPT'$.
\end{proof}
	
\begin{theorem}
	Algorytm jest $3$-przybliżony dla problemu najkrótszego wspólnego nadsłowa
\end{theorem}

\begin{proof}
	Każde słowo $s_i$ jest podsłowem pewnego $\sigma(c_j)$, ponieważ $\mathbb{C}$ jest pokryciem cyklowym w $G$. Zatem rozwiązanie jest dopuszczalne.

	Z Lematu 1 wynika, że $\sum |\sigma(c_i)| - |\sigma| \ge 0.5 (\sum |\sigma(c_i)| - |\sigma_{OPT}|)$, a więc $|\sigma| \le 0.5 (\sum |\sigma(c_i)| - |\sigma_{OPT}|) + |\sigma_{OPT}| = 0.5 \sum overlap(r_i, r_{i + 1}) + |\sigma_{OPT}| \le \sum |\alpha(c_i)| + |\sigma_{OPT}|$.
	
	Niech $r_{OPT}$ będzie najkrótszym nadsłowem $r_1$, $r_2$, \ldots. Ponieważ $r_i$ występuje na początku i na końcu $\sigma(c_i)$, to $\sum |\sigma(c_i)| - |\sigma_{OPT}| \ge \sum |r_i| - |r_{OPT}|$. Ponadto, $\sum |\sigma(c_i)| = |\alpha(c_i)| + \sum |r_i|$, więc $|\sigma_{OPT}| \le |\alpha(c_i)| + |r_{OPT}|$.
	
	$APX = |\sigma| \le \sum |\alpha(c_i)| + |\sigma_{OPT}| \le 2 \sum |\alpha(c_i)| + |r_{OPT}| \le 3 OPT$.
\end{proof}
